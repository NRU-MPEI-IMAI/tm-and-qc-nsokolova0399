\documentclass[a4paper]{article}
\usepackage[14pt]{extsizes} % для того чтобы задать нестандартный 14-ый размер шрифта
\usepackage[utf8]{inputenc}
\usepackage[russian]{babel}
\usepackage{setspace,amsmath}
\usepackage{graphicx}
\usepackage{listings}
\usepackage[left=20mm, top=15mm, right=15mm, bottom=15mm, nohead, footskip=10mm]{geometry} % настройки полей документа
 
\begin{document} % начало документа
 % НАЧАЛО ТИТУЛЬНОГО ЛИСТА
\begin{center}
\hfill \break
\large{Министерство образования и науки РФ}\\
\large{ФЕДЕРАЛЬНОЕ ГОСУДАРСТВЕННОЕ БЮДЖЕТНОЕ ОБРАЗОВАТЕЛЬНОЕ УЧРЕЖДЕНИЕ}\\ 
\large{ВЫСШЕГО ОБРАЗОВАНИЯ}\\
\large{НАЦИОНАЛЬНЫЙ ИССЛЕДОВАТЕЛЬСКИЙ УНИВЕРСИТЕТ «МЭИ»}\\
\hfill \break
\normalsize{Прикладная математика и информатика}\\
 \hfill \break
\normalsize{Кафедра прикладной математики и искусственного интеллекта}\\
\hfill\break
\hfill \break
\hfill \break
\hfill \break
\large{Теоретические модели вычисления}\\
\hfill \break
\large{Домашнее задание №3}\\
\large{Машины Тьюринга и квантовые вычисления}\\
\hfill \break
\hfill \break
\hfill \break
\hfill \break
\hfill \break
\hfill \break
\end{center}
 
\normalsize{ 
\begin{flushright}
\ Преподаватель: Ивлиев С. А. \\
\ Студент: Соколова А.С. \
\end{flushright}
\hfill \break
\hfill \break
\hfill \break
\hfill \break
\hfill \break
\begin{center} Москва 2022 \end{center}
\thispagestyle{empty} % выключаем отображение номера для этой страницы
} 
% КОНЕЦ ТИТУЛЬНОГО ЛИСТА

% СОДЕРЖАНИЕ
\newpage
    \tableofcontents % Вывод содержания
\newpage
% КОНЕЦ СОДЕРЖАНИЯ 
 

\newpage
\section{Машины Тьюринга}

Работу требуется выполнять в системе \url{turingmachine.io}. \\\\
Для сдачи заданий 1-2 требуется прикрепить файлы YAML с исходным кодом проекта. Каждый файлы должен иметь наименование задание\_пункт.yml, к примеру 1\_1.yml для первой задачи первого задания. \\\\

% ЗАДАНИЕ 1
\subsection{Операции с числами}

Реализуйте машины Тьюринга, которые позволяют выполнять следующие операции:
\begin{enumerate}
    \item Сложение двух унарных чисел (1 балла)
    \begin{figure}[h]
        \centering
        \includegraphics[width=17cm]{1.1.jpg}
    \end{figure}
\\
\ Используем дополнительный символ x \\
\ Считаем первую единицу, заменяем ее  на x. Идем до конца, считываем пробел и заменяем его на единицу. Идем влево до x. Считываем x, заменяем его на пробел. \\
\ Каждую следующую единицу по очереди заменяем на x, копируем единицу в конец и т д. \\
\ 1+1 -> 11\\
\ 111+1 -> 1111\\
\ 1+11111 -> 111111\\
\\
\begin{tabular}{|*{5}{c|}}
\textbf{ } & 1 & + & x & ' '\\
\hline\hline
q0 & <q1,x,R> & <done,' ',R> & & <done, ' ',R> \\
\hline\hline
q1 & <q1,1,R> & <q2,+,R> & & \\
\hline\hline
q2 & <q2,1,R> & & & <' ',1,L> \\
\hline\hline
q3 & <q3,1,R> & <q4,+,L> & & \\
\hline\hline
q4 & <q4,1,L> & &<q0,' ', R> & \\
\hline\hline
\end{tabular}
    \item Умножение унарных чисел (1 балл)
        \begin{figure}[h]
        \centering
        \includegraphics[width=17cm]{1.2.jpg}
    \end{figure}
    
\ 1*1 -> 1\\
\ 111*1 -> 111\\
\ 1*11111 -> 11111\\
\\
\begin{tabular}{|*{4}{c|}}
\textbf{ } & 1 & * & ' '\\
\hline\hline
q0 & <q1,' ',R> & <skip,*,R> & \\
\hline\hline
q1 & <q1,1,R> & <q2,*,R> & \\
\hline\hline
q2 & <q3,' ',R> & <back,' ',L> &\\
\hline\hline
q3 & <q3,1,R> & &<result,' ',R>\\
\hline\hline
result & <result,1,R> & &<q4,1,L>\\
\hline\hline
q4 & <q4,1,L>& <q5,*,R> &<q4,' ',L> \\
\hline\hline
q5 & <q5,1,R>& &<q2,1,R> \\
\hline\hline
back & <back,1,L>& <back,*,L> &<q0,1,R> \\
\hline\hline
skip & <skip,1,R>& &<done,' ',R> \\
\end{tabular}
\\
\\
\ Рассотрим подробнее на простом примере 11*1\\
\ 1. Переходим в состояние q0 по 1 и заменяем 1 на пробел и передвигаемся правее. Получаем промежуточный результат: \\
\begin{tabular}{|*{2}{c|}}
\textbf{ } & 1 \\
\hline\hline
q0 & <q1,' ',R> \\
\end{tabular}
\\ \\
\ \_1*1 \\
\ 2. Считываем следующую единицу, переходим в состояние q1 по 1. Считываем единицы и передвигаемся правее пока не встретим *. То есть переходим в состояние q2 по *. Промежуточный результат без изменения, только головка передвигается: \\
\begin{tabular}{|*{3}{c|}}
\textbf{ } & 1 &*\\
\hline\hline
q1 & <q1,1,R> & <q2,*,R> \\
\end{tabular}
\\ \\
\ \_1*1 \\
\ 3. Далее после знака * встречаем единицу. То есть переходим в состояние q2 по 1 и заменяем 1 на пробел. Далее попадаем в состояние q3: \\
\begin{tabular}{|*{2}{c|}}
\textbf{ } & 1 \\
\hline\hline
q2 & <q3,' ',R>\\
\end{tabular}
\\ \\
\ \_1*\_ \\
\ 4. Считываем еще один пробел в состоянии q3 и переходим в состояние result: \\
\begin{tabular}{|*{2}{c|}}
\textbf{ } & ' ' \\
\hline\hline
q3 & <result,' ',R>\\
\end{tabular}
\\ \\
\ \_1*\_\_\_ \\
\ 5. Так как текущая ячейка пробел, то из состояния result по пробелу переходим в состояние q4, заменяя пробел на единицу и передвигая головку левее: \\
\begin{tabular}{|*{2}{c|}}
\textbf{ } & ' ' \\
\hline\hline
result &<q4,1,L>\\
\end{tabular}
\\ \\
\ \_1*\_\_1\\
\ 6. Из состояния q4 по пробелу и единице двигаемся левее, пропуская их: \\
\begin{tabular}{|*{3}{c|}}
\textbf{ }  & 1 & ' ' \\
\hline\hline
q4 & <q4,1,L>&<q4,' ',L> \\
\end{tabular}
\\ \\
\ \_1*\_\_1\\
\ 7. Далее встречаем умножение * и переходим в состояние q5 двигаясь правее. А из состояния q5 встрречая пробел, заменяем его на  и переходя в состояние q2 правее: \\
\begin{tabular}{|*{2}{c|}}
\textbf{ }  & * \\
\hline\hline
q4 & <q5,*,R> \\
\end{tabular}\\ \\
\begin{tabular}{|*{2}{c|}}
\textbf{ }  & ' ' \\
\hline\hline
q5 & <q2,1,R> \\
\end{tabular}
\\ \\
\ \_1*1\_1\\
\ 8. Из состояния q2 по пробелу переходим в состояние back двигаемся правее. Затем из состояния back по единице и умножению двигаемся левее, переходя к начальному состоянию. Когда встречаем пробел, заменяем его на единицу: \\
\begin{tabular}{|*{4}{c|}}
\textbf{ }  & 1 & * & ' ' \\
\hline\hline
back & <back,1,L>& <back,*,L> &<q0,1,R> \\
\end{tabular}
\\ \\
\ 11*1\_1\\

\ 9. Следующую единицу снова заменяем на пробел и т д : \\
\begin{tabular}{|*{2}{c|}}
\textbf{ }  & 1 \\
\hline\hline
q0 & <q1,' ',R> \\
\end{tabular}
\\ \\
\ 1\_*1\_1\\
10. \ 1\_*\_\_1\\
11. \ 1\_*\_\_11\\
12. \ 1\_*1\_11\\
13. \ 11*1\_11\\
14. \ Передвигаем головку на начало результата\\
\begin{figure}[h]
        \centering
        \includegraphics[width=7cm]{1.1screen.jpg}
    \end{figure}
\end{enumerate}
% КОНЕЦ ЗАДАНИЯ 1

% НАЧАЛО ЗАДАНИЯ 2
\subsection{Операции с языками и символами}

Реализуйте машины Тьюринга, которые позволяют выполнять следующие операции:
\begin{enumerate}
    \item Принадлежность к языку $L = \{ 0^n1^n2^n \}, n \ge 0$ (0.5 балла)
    \begin{figure}[h]
        \centering
        \includegraphics[width=16cm]{2.1.jpg}
    \end{figure}
    \\
\ Примеры данных, при которых МТ корректно завершает работу:\\
\ input: '001122' \\
\ Как выглядит на ленте завершение программы:\\
\begin{figure}[h]
        \centering
        \includegraphics[width=7cm]{2.1screen1.jpg}
\end{figure}
\\
\ input: '012' \\
\ input: '001122' \\
\ input: '000111222' \\
\ input: '000011112222' \\
\ Примеры данных, при которых МТ некорректно завершает свою работу:\\
\ input: '00011222'\\
\ Как выглядит на ленте завершение программы:\\
\begin{figure}[h]
        \centering
        \includegraphics[width=7cm]{2.1screen.jpg}
\end{figure}
\\
\ input: '000122'\\
\ input: '221100'\\
\ input: '1122'\\
\ input: '00122'\\
\ input: '00112'\\
\ input: '3'\\
\ input: '000111'\\
\ input: '01122'\\
\\ \newpage
    \item Проверка соблюдения правильности скобок в строке (минимум 3 вида скобок) (0.5 балла)
\\
\begin{figure}[h]
        \centering
        \includegraphics[width=17cm]{2.2.jpg}
\end{figure}\\ 
\ Примеры данных, при которых МТ корректно завершает работу:\\
\ input: '$($ $[$ $($ $[$ $]$ $)$ $]$ $)$' \\
\ input: '$($ $)$ $[$ $]$' \\
\ input: '$($ $)$ $[$ $]$ \{ \}' \\
\ input: '$($ $($ )$ $[$ $]$ \{\}$)' \\
\ Примеры данных, при которых МТ некорректно завершает свою работу:\\
\ input: '$($ $[$ $($ $]$ $)$'\\
\ input: '$($ $($ \{ $[$ $]$ \} $($ $[$ $]$ $)$'\\
\ input: '$($ $($ $($ $)$ $)$'\\
\ input: '$($'\\
\\
\newpage
    \item Поиск минимального по длине слова в строке (слова состоят из символов 1 и 0 и разделены пробелом) (1 балл)
   \begin{figure}[h]
        \centering
        \includegraphics[width=17cm]{2.3.jpg}
   \end{figure}\\ 
\ Тесты:\\
\ input: '10101 101 100' \\
\ Результат:\\
   \begin{figure}[h]
        \centering
        \includegraphics[width=3cm]{2.3screen.jpg}
   \end{figure}\\ 
\ input: '11 01 10'\\
\ Результат:
   \begin{figure}[h]
        \centering
        \includegraphics[width=3cm]{2.3screen1.jpg}
   \end{figure}\\ 
\\ \\
\ input: '1 101 110'\\
\ Результат:\\
   \begin{figure}[h]
        \centering
        \includegraphics[width=2cm]{2.3screen2.jpg}
   \end{figure}\\ 
\ input: '1'\\
\ Результат:\\
   \begin{figure}[h]
        \centering
        \includegraphics[width=2cm]{2.3screen2.jpg}
   \end{figure}\\ 
\end{enumerate}\\
% КОНЕЦ ЗАДАНИЯ 2


% НАЧАЛО ЗАДАНИЯ 3
\section{Квантовые вычисления}

\subsection{Генерация суперпозиций 1 (1 балл)}

Дано $N$ кубитов ($1 \le N \le 8$) в нулевом состоянии $\Ket{0\dots0}$. Также дана некоторая последовательность битов, которое задаёт ненулевое базисное состояние размера $N$. Задача получить суперпозицию нулевого состояния и заданного.

$$\Ket{S} = \frac{1}{\sqrt2}(\Ket{0\dots0} +\Ket{\psi})$$

То есть требуется реализовать операцию, которая принимает на вход:

\begin{enumerate}
    \item Массив кубитов $q_s$
    \item Массив битов $bits$ описывающих некоторое состояние $\Ket{\psi}$. Это массив имеет тот же самый размер, что и $qs$. Первый элемент этого массива равен $1$.
\end{enumerate}


Заготовка для кода:
\begin{lstlisting}
namespace Solution {
        open Microsoft.Quantum.Primitive;
        open Microsoft.Quantum.Canon;
        operation Solve (qs : Qubit[], bits : Bool[]) : ()
        {
            body
            {

            }
        }
}
\end{lstlisting}

\subsection{Различение состояний 1 (1 балл)}

Дано $N$ кубитов ($1 \le N \le 8$), которые могут быть в одном из двух состояний:

$$\Ket{GHZ} = \frac{1}{\sqrt2}(\Ket{0\dots0} +\Ket{1\dots1})$$
$$\Ket{W} = \frac{1}{\sqrt N}(\Ket{10\dots00}+\Ket{01\dots00} + \dots +\Ket{00\dots01})$$

Требуется выполнить необходимые преобразования, чтобы точно различить эти два состояния. Возвращать $0$, если первое состояние и 1, если второе. 
\\\\
Заготовка для кода:
\begin{lstlisting}
namespace Solution {
        open Microsoft.Quantum.Primitive;
        open Microsoft.Quantum.Canon;
        operation Solve (qs : Qubit[]) : Int
        {
            body
            {

                return 
            }
        }
}
\end{lstlisting}

% КОНЕЦ ЗАДАНИЯ 3

% НАЧАЛО ЗАДАНИЯ 4

% КОНЕЦ ЗАДАНИЯ 4

% НАЧАЛО ЗАДАНИЯ 5

% КОНЕЦ ЗАДАНИЯ 5
\end{document}  % КОНЕЦ ДОКУМЕНТА !